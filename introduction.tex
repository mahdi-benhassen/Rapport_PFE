
\chapter*{Introduction générale}
\addcontentsline{toc}{chapter}{Introduction générale}
  Le transport est l’une des choses fondamentales dans la vie quotidienne. Les moyens de transport sont divers, on peut distinguer le transport routier, ferroviaire, aérien, et maritime. Le plus abondant et couramment utilisé entre eux est le transport routier comme les voitures, les motos, les autobus, les vélos. Les moyens de transport sont provoqués des problèmes ardus tels que : l’embouteillage, le taux de consommation des carburants élevé, pollution de l’environnement et les accidents routiers. Par exemple, les embouteillages coutent 178 euros par seconde, soit plus de 5,6 milliards d’euros chaque année seulement en France, dont 563 millions d’euros attribuables à la diffusion de carbone, 3,5 milliards d’euros liés au temps perdu et 1,8 milliard de faits de l’augmentation des prix de la consommation en conséquence des pertes de productivité [1]. Les accidents routiers conduisent également à des dégâts énormes. Les statistiques montrent que 1.3 million de personnes perdent la vie à cause des accidents routiers chaque année. Le rythme actuel prévoira que ce nombre de décès va s’augmenter et arrivera à 2.4 millions de personnes qui tueront sur les routes en 2030. Ajouté à cela, les pertes économiques dues aux accidents routiers qui variées de 65 à 100 milliards de dollars chaque année [1]. Dans la majorité des cas, elles peuvent être causées par un manque de discipline du conducteur, le refus de suivre les règles de la circulation, ou de l’inefficacité des infrastructures. Nous pourrions admettre que c’est vraiment difficile de prévoir les accidents routiers.

\begin{figure}[h]
\centering
\includegraphics[width=0.8\textwidth]{images/image3.jpeg}
\caption{Image extracted}
\end{figure}
\begin{figure}[h]
\centering
\includegraphics[width=0.8\textwidth]{images/image4.jpeg}
\caption{Image extracted}
\end{figure}


\begin{figure}[h]
\centering
\includegraphics[width=0.8\textwidth]{images/image5.jpeg}
\caption{Image extracted}
\end{figure}


\begin{figure}[h]
\centering
\includegraphics[width=0.8\textwidth]{images/image6.jpeg}
\caption{Image extracted}
\end{figure}


Figure 1: Gravités des accidents routiers

  Les progrès de la technologie de la sécurité routière nous donnent un énorme potentiel pour réduire ces dégâts, particulièrement le nombre d’accidents et leur gravité quand ils se produisent. Il y a quelque année des systèmes de transports intelligents (Intelligent Transportation Systems - ITS) créé des applications qui ont fait leurs preuves de sauver des vies dans les routes (Safety Life Applications) [2] [3]. La technologie VANET (Vehicular Ad hoc Networks) est une approche prometteuse pour faciliter la sécurité routière, la gestion du trafic et la diffusion d’infotainment pour les conducteurs et les passagers. Nous attendons un avantage important pour la sécurité de systèmes de transport coopératifs intelligents basés sur une technologie émergente où les véhicules communiquent entre eux et à l’infrastructure routière pour aider à éviter les accidents.

  Pour implémenter la technologie VANET, il doit charger de déterminer et respecter toutes les mesures de sécurité appropriées, l’environnement, la santé et la protection contre les interférences et toutes les lois et les règlementations applicables. L’un des buts ultimes de la conception d’un tel réseau est de résister à divers abus et aux attaques malveillantes de sécurité. Nous avons focalisé dans ce projet de fin d’études sur les moyens de résoudre ce problème majeur de sécurité pour les massages de sécurité de base (Basic Safety Message). Nous avons donné une description générale de services de sécurité utilisée pour les applications VANET et leur utilisation. Aussi, nous avons fait une implémentation hardware d’algorithme de hachage sécurise (Secure Hash Algorithm) SHA-256 utiliser dans la norme IEEE1609.2 pour la signature numérique des messages BSM (Basic Safety Message).


\chapter*{  Dans ce projet de fin d’études, après l’introduction générale nous nous destinerons le premier chapitre pour introduire le processus de normalisation en cours, qui couvre la technologie VANET, et de spécifie le codage et la structure des messages de sécurité de base (BSM : Basic Safety Messgae) générés et consommés par les applications de VANET. Nous avons étudierons aussi un cas d’application de réseau VANET avec le scénario d’un feu de freinage d’urgence électronique.}
\addcontentsline{toc}{chapter}{  Dans ce projet de fin d’études, après l’introduction générale nous nous destinerons le premier chapitre pour introduire le processus de normalisation en cours, qui couvre la technologie VANET, et de spécifie le codage et la structure des messages de sécurité de base (BSM : Basic Safety Messgae) générés et consommés par les applications de VANET. Nous avons étudierons aussi un cas d’application de réseau VANET avec le scénario d’un feu de freinage d’urgence électronique.}
  Nous aborderons ensuite dans le chapitre2 les normes pratiques et les mécanismes de sécurité introduits pour VANET et les communications DSRC pour s’authentifier des certificats au but d’atteindre le niveau de sécurité approprié des messages BSM. Nous avons spécifierons ainsi l’algorithme de hachage sécurisé SHA-256.

  Dans le chapitre3, nous utiliserons le logiciel de synthèse haut niveau GAUTT pour la spécification d’algorithme de compresseur de SHA-256 jusqu’à niveau RTL et graphe de flux des données (Data Flow Graph) et nous introduirons cette approche de conception avec les diffèrent étapes.

  Dans le chapitre4, nous détaillerons le design RTL manuel et les améliorations qui se font sur la conception de base de SHA-256 et les étapes d’implémentation sur FPGA avec l’utilisation des outils de synthèse tels que XILINX ISE et XST.

  Enfin, nous donnerons une conclusion aux résultats obtenus et les perspectives d’améliorer ce travail et  la sécurité des réseaux VANET en général.

